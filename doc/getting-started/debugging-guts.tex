\section{Debugging Designs}

Chisel provides a number of mechanisms for debugging your designs.  
A designer can format and print out strings to display signals over time.
Runtime assertions can be declared with the \verb+assert+ construct.
Circuits can be displayed with the dot backend.
Complete signal values can be dumped over time in VCD format.
Finally, circuits can be interacted with the debug API.

\section{Printf and Sprintf}

Chisel provides the ability to format and print strings for debugging
purposes.  The \code{printf} and \code{sprintf} construct are similar to their
C namesakes: they take a format string and a variable number of arguments,
then print or return a string, respectively.  During simulation, \code{printf}
prints the formatted string to the console on rising clock edges.
\code{sprintf}, on the other hand, returns the formatted string as a bit
vector.

Supported format specifiers are \code{\%b} (binary number), \code{\%d}
(decimal number), \code{\%x} (hexadecimal number), and \code{\%s} (string
consisting of a sequence of 8-bit extended ASCII characters).  (\code{\%\%}
specifies a literal \code{\%}.) Unlike in C, there are no width modifiers: the
bit width of the corresponding argument determines the width in the string
representation.

The following example prints the line \code{"0x4142 16706 AB"} on cycles when
\code{c} is true:

\begin{scala}
val x = Bits(0x4142)
val s1 = sprintf("%x %s", x, x);
when (c) { printf("%d %s\n", x, s1); }
\end{scala}

\section{Assert}

Runtime assertions are provided by the \code{assert} construct.  During
simulation, if an assertion's argument is false on a rising clock edge,
an error is printed and simulation terminates.  For example, the following
will terminate simulation after ten clock cycles:

\begin{scala}
val x = Reg(init = UInt(0, 4))
x := x + UInt(1)
assert(x < UInt(10))
\end{scala}

\section{Circuit Graph Visualization}

In Chisel, circuits are constructed using a textual program.
Sometimes it is useful to see the constructed circuit graph.
Users can produce at dot graph file viewable by vizgraph using the dot backend.

\todo{finish examples below}

\begin{scala}
chiselMain ... --backend dot ...
\end{scala}

\begin{bash}
vizgraph -pdf test.dot
\end{bash}

\section{VCD Dumps}

One powerfully way of debugging is by saved the values of all signals and state over time.
Chisel supports this by running circuits under test while 
dumping \verb+VCD+ (Value Change Dump) files using the \verb+--vcd+ option:

\begin{bash}
chiselMainTest ... --vcd ...
\end{bash}

These VCD files can then viewed using either commercial waveform viewers like \verb+VCS+ or open source ones like \verb+gtkwave+.

\todo{picture of waveform}

% \section{Debug API}
% 
% For those that want lower level control over their designs, 
% they can drive their circuits through a low level debug API.
