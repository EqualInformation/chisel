\documentclass[10pt,landscape]{article}
\usepackage{multicol}
\usepackage[landscape]{geometry}
\usepackage[procnames]{listings}
\usepackage[parfill]{parskip}
\usepackage{fixltx2e}

% "define" Scala
\usepackage[T1]{fontenc}  
\usepackage[scaled=0.82]{beramono}  
\usepackage{microtype} 

\sbox0{\small\ttfamily A}
\edef\mybasewidth{\the\wd0 }

\lstdefinelanguage{scala}{
  morekeywords={abstract,case,catch,class,def,%
    do,else,extends,false,final,finally,%
    for,if,implicit,import,match,mixin,%
    new,null,object,override,package,%
    private,protected,requires,return,sealed,%
    super,this,throw,trait,true,try,%
    type,val,var,while,with,yield},
  sensitive=true,
  morecomment=[l]{//},
  morecomment=[n]{/*}{*/},
  morestring=[b]",
  morestring=[b]',
  morestring=[b]"""
}

\usepackage{color}
\definecolor{dkgreen}{rgb}{0,0.6,0}
\definecolor{gray}{rgb}{0.5,0.5,0.5}
\definecolor{mauve}{rgb}{0.58,0,0.82}

% Default settings for code listings
\lstset{language=scala,
  showstringspaces=false,
  columns=fixed, % basewidth=\mybasewidth,
  basicstyle={\small\ttfamily},
  numbers=none,
  numberstyle=\footnotesize\color{gray},
  % identifierstyle=\color{red},
  keywordstyle=\color{blue},
  commentstyle=\color{dkgreen},
  stringstyle=\color{mauve},
  breakatwhitespace=true,
  procnamekeys={def, val, var, class, trait, object, extends},
  procnamestyle=\ttfamily\color{red},
}

\lstnewenvironment{scala}
{\lstset{language=scala}}
{}
\lstnewenvironment{cpp}
{\lstset{language=C++}}
{}
\lstnewenvironment{bash}
{\lstset{language=bash}}
{}
\lstnewenvironment{verilog}
{\lstset{language=verilog}}
{}

\newcommand{\isc}{
\lstinline
}

\lstdefinestyle{scala}{language=scala,
  showstringspaces=false,
  columns=fixed, % basewidth=\mybasewidth,
  basicstyle={\small\ttfamily},
  numbers=none,
  numberstyle=\footnotesize\color{gray},
  % identifierstyle=\color{red},
  keywordstyle=\color{blue},
  commentstyle=\color{dkgreen},
  stringstyle=\color{mauve},
  breakatwhitespace=true,
  procnamekeys={def, val, var, class, trait, object, extends},
  procnamestyle=\ttfamily\color{red},
}


% Remove section numbering
\setcounter{secnumdepth}{0}

\geometry{top=1cm,left=1cm,right=1cm,bottom=1cm}

\pagestyle{empty}

\makeatletter
\renewcommand{\section}{\@startsection{section}{1}{0mm}%
                                {-0.ex plus 1 ex}%
                                {0.01ex plus .01ex}%x
                                {\normalfont\large\bfseries}}
\renewcommand{\subsection}{\@startsection{subsection}{2}{0mm}%
                                {-0.ex plus 1 ex}%
                                {0.01ex plus .01ex}%
                                {\normalfont\normalsize\bfseries}}
\renewcommand{\subsubsection}{\@startsection{subsubsection}{3}{0mm}%
                                {-0.ex plus 1 ex}%
                                {0.01ex plus .01ex}%
                                {\normalfont\small\bfseries}}
\makeatother




\begin{document}
\begin{multicols}{3}

\begin{center}
\Large{Chisel Cheat Sheet}
\end{center}

\renewcommand{\tabcolsep}{.5mm}

\section{Notation}
\verb$c, x, y$ are Chisel \verb$Data$s; \verb$n, m$ are Scala \verb$Int$s \newline
\verb$wx, wy$ are the widths of \verb$x, y$ (respectively) \newline
\verb$minVal(x)$, \verb$maxVal(x)$ are the min or max of \verb$x$ \newline
\verb$s$ is a Scala string; \verb$b$ is a Scala boolean \newline
\verb$[ ... ]$ in functions are optional arguments

\section{Basic Operators}
\begin{tabular}{l l}
\verb$val x = UInt()$ & Allocate \verb$a$ as wire of \verb$UInt()$ \\
\verb$x := y$ & Assign \verb$y$ to wire \verb$x$ \\
\verb$x <> y$ & Connect \verb$x$ and \verb$y$, inferring direction \\
\end{tabular}

\section{Basic Data Types}
\subsection{Bool Constructors}
\verb$Bool([x:Boolean])$ \newline
\begin{tabular}{l l l}
& \verb$x$ & create a literal from Boolean \\
& & or declare unassigned if missing \\
\end{tabular}

\subsection{Bits, SInt, UInt Constructors}
\verb$Bits(x:Int|String, width:Int)$ \newline
\verb$UInt(x:Int|String, width:Int)$ \newline
\verb$SInt(x:Int|String, width:Int)$ \newline
\begin{tabular}{l l l}
& \verb$x$ & create a literal from \verb$Int$ or parsed \verb$String$ \\
& & or declare unassigned if missing \\
& \verb$width$ & bit width (or inferred) \\
\end{tabular}

\subsection{Bits, UInt, SInt Casts}
\begin{tabular}{l l}
\verb$UInt$ $\rightarrow$ \verb$SInt$ & Zero-extend to SInt \\
(all others) & Reinterpret cast \\
\end{tabular}

\subsection{Bool Operators}
\begin{tabular}{l l l}
Chisel & Explanation & Width \\
\hline
\hline
\verb$!x$ & Logical NOT & \verb$1$ \\
\verb$x && y$ & Logical AND & \verb$1$ \\
\verb$x || y$ & Logical OR & \verb$1$ \\
\end{tabular}

\subsection{Bits Operators}
\begin{tabular}{l l l}
Chisel & Explanation & Width \\
\hline
\hline
\verb$x(n)$ & Extract bit (0 is LSB) & \verb$1$ \\
\verb$x(n, m)$ & Extract bitfield & \verb$n - m + 1$ \\
\verb$x << y$ & Left shift & \verb$wx + maxVal(y)$ \\
\verb$x >> y$ & Right shift & \verb$wx - minVal(y)$ \\
\verb$x << n$ & Left shift & \verb$wx + n$ \\
\verb$x >> n$ & Right shift & \verb$wx - n$ \\
\verb$Fill(n, x)$ & Replicate \verb$x$ \verb$n$ times & \verb$n * wx$ \\
\verb$Cat(x, y)$ & Concatenate bitfields & \verb$wx + wy$ \\
\verb$Mux(c, x, y)$ & If \verb$c$ then \verb$x$ else \verb$y$ & \verb$max(wx, wy)$ \\
\hline
\verb$~x$ & Bitwise NOT & \verb$wx$ \\
\verb$x & y$ & Bitwise AND & \verb$max(wx, wy)$ \\
\verb$x | y$ & Bitwise OR & \verb$max(wx, wy)$ \\
\verb$x ^ y$ & Bitwise XOR & \verb$max(wx, wy)$ \\
\hline
\verb$x === y$ & Equality & \verb$1$ \\
\verb$x != y$ & Inequality & \verb$1$ \\
\hline
\verb$andR(x)$ & AND-reduce & \verb$1$ \\
\verb$orR(x)$ & OR-reduce & \verb$1$ \\
\verb$xorR(x)$ & XOR-reduce & \verb$1$ \\
\end{tabular}

\subsection{UInt, SInt Operators}
Bitwidths only valid for UInt operations

\begin{tabular}{l l l}
Chisel & Explanation & Width \\
\hline
\hline
\verb$x + y$ & Addition & \verb$max(wx, wy)$ \\
\verb$x - y$ & Subtraction & \verb$max(wx, wy)$ \\
\verb$x * y$ & Multiplication & \verb$wx + wy$ \\
\verb$x / y$ & Division & \verb$wx$ \\
\verb$x % y$ & Modulus & \verb$bits(maxVal(y) - 1)$ \\
\hline
\verb$x > y$ & Greater than & \verb$1$ \\
\verb$x >= y$ & Greater than or equal & \verb$1$ \\
\verb$x < y$ & Less than & \verb$1$ \\
\verb$x <= y$ & Less than or equal & \verb$1$ \\
\hline
\verb$x >> y$ & Arithmetic right shift & \verb$wx - minVal(y)$ \\
\verb$x >> n$ & Arithmetic right shift & \verb$wx - n$ \\
\end{tabular}

\section{Helpers}
\subsection{When}
Use \verb$when$ to execute statements conditionally \newline
\verb$when$ behaves similarly to Verilog \newline \verb$always @(posedge clk)$
\begin{scala}
when(condition1) {
  // run if condition1 true
} .elsewhen(condition2) {
  // run if condition2 true
} .unless(condition3) {
  // run if condition3 false
} .otherwise {
  // run if none of the above true
}
\end{scala}

\subsection{Switch}
Use \verb$switch$ to execute statements conditionally \newline
on the value of a wire \newline
\begin{scala}
switch(x) {
  is(value1) {
    // run if x === value1
  } is(value2) {
    // run if x === value2
  }
}
\end{scala}

\subsection{Enum}
Use \verb$enum$s to generate list values \newline
\verb$val s1::s2::$ ... \verb$::sn::Nil$ \newline
\verb$    := Enum(nodeType:UInt, n:Int)$ \newline
\begin{tabular}{l l l}
& \verb$s1$, \verb$s2$, ..., \verb$sn$ & will be created with distinct values \\
& \verb$nodeType$ & type of \verb$s1$, \verb$s2$, ..., \verb$sn$ \\
& \verb$n$ & element count \\
\end{tabular}

\section{Aggregate Types}
\subsection{Bundle}
\verb$Bundle$ contains \verb$Data$ types indexed by name
\subsubsection{Defining}
Define a \verb$Bundle$ by subclassing \verb$Bundle$ and \newline
populating with components:
\begin{scala}
class MyBundle extends Bundle {
  val a = Bool()
  val b = UInt(width = 32)
}
\end{scala}
\subsubsection{Constructor}
Instantiate a \verb$Bundle$ subclass: \newline
\verb$val my_bundle = new MyBundle()$ \newline
Or define an inline Bundle type:
\begin{scala}
val my_bundle = new Bundle {
  val a = Bool()
  val b = UInt(width = 32)
}
\end{scala}
\subsubsection{Using}
Access elements using dot notation: \newline
\verb$val bundleVal = my_bundle.a$ \newline
\verb$my_bundle.a := Bool(true)$

\subsection{Vec}
\verb$Vec$ is an indexable vector of \verb$Data$ types
\subsubsection{Constructor}
\verb$val myVec = Vec.fill(n:Int) {gen:Data}$ \newline
\begin{tabular}{l l l}
& \verb$n$ & vector depth (elements) \\
& \verb$gen$ & element data type \\
\end{tabular}
\subsubsection{Using}
Access elements by indexing into a \verb$Vec$: \newline
\verb$readVal := myVec(ind:Data)$ (index by wire) \newline
\verb$readVal := myVec(idx:Int)$ (static index) \newline
\verb$myVec(ind:Data) := writeVal$ (index by wire) \newline
\verb$myVec(idx:Int) := writeVal$ (static index) \newline

\section{State Elements}
\subsection{Registers}
\subsubsection{Constructor}
\verb$val my_reg = Reg([outType:Data], [next:Data],$ \newline
\verb$                 [init:Data])$ \newline
\begin{tabular}{l l l}
& \verb$outType$ & register type (or inferred) \\
& \verb$next$ & update value every clock \\
& \verb$init$ & initialization value on reset \\
\end{tabular}
\subsubsection{Updating}
Assign to latch a new value on the next clock: \newline
\verb$my_reg := next_val$ \newline
The last update (lexically, per clock) runs

\subsection{Read-Write Memory}
\subsubsection{Constructor}
\verb$val my_mem = Mem(out:Data, n:Int,$ \newline
\verb$                 seqRead:Boolean)$ \newline
\begin{tabular}{l l l}
& \verb$out$ & memory element type \\
& \verb$n$ & memory depth (elements) \\
& \verb$seqRead$ & only update reads on clock edge \\
\end{tabular}
\subsubsection{Usage}
Access elements by indexing into a \verb$Mem$: \newline
\verb$val readVal = mem(addr:UInt|Int)$ \newline
Synchronous read: assign output to \verb$Reg$ \newline
\verb$mem(addr:UInt|Int) := y$

\section{Modules}
\subsubsection{Defining}
Define a \verb$Module$ by subclassing \verb$Module$ and \newline
populating with components and code: \newline
(\verb$Module$s may be parameterized and hierarchical) \newline
\begin{scala}
class SyncAdder(width:Int) extends Module {
  val io = new Bundle {
    val op_a = UInt(INPUT,  width)
    val op_b = UInt(INPUT,  width)
    val out  = UInt(OUTPUT, width)
  }
  val latch = new Reg(UInt())
  latch := io.op_a + io.op_b
  out := io.latch
}
\end{scala}
\subsubsection{Usage}
Elements can be accessed using dot notation: \newline
\begin{scala}
val my_module = Module(new SyncAdder(32))
my_module.io.op_a := UInt(1)
my_module.io.op_b := some_data
val sum := my_module.io.out
\end{scala}

\section{Hardware Generation}
\subsection{Functions}
\subsubsection{Defining}
Define Scala functions containing Chisel code: \newline
\begin{scala}
def FunAdder(op_a:UInt, op_b:UInt): UInt {
  a + b
}
\end{scala}
\subsubsection{Usage}
Hardware is instantiated when called: \newline
\begin{scala}
sum := FunAdder(UInt(1), some_data)
\end{scala}

\subsection{If, For}
Use Scala \verb$if$, \verb$for$ to generate hardware

\section{Standard Library}
\subsection{Decoupled}
\verb$Decoupled$ is a \verb$Bundle$ with a ready-valid interface,\newline
where the consumer should flip the interface: \newline
\subsubsection{Constructor}
\verb$val my_rv = Decoupled(type:Data)$ \newline
\begin{tabular}{l l l}
& \verb$type$ & data type \\
\end{tabular}
\subsubsection{Interface}
\begin{tabular}{l l l}
& \verb$.ready$ & ready \verb$Bool$ \\
& \verb$.valid$ & valid \verb$Bool$ \\
& \verb$.bits$ & data \\
\end{tabular}

\subsection{PipeIO}
\verb$PipeIO$ is a \verb$Bundle$ with a valid interface,\newline
where the consumer should flip the interface: \newline
\subsubsection{Constructor}
\verb$val my_rv = PipeIO(type:Data)$ \newline
\begin{tabular}{l l l}
& \verb$type$ & data type \\
\end{tabular}
\subsubsection{Interface}
\begin{tabular}{l l l}
& \verb$.valid$ & valid \verb$Bool$ \\
& \verb$.bits$ & data \\
\end{tabular}

\subsection{Arbiters}
\verb$Arbiter$ is a \verb$Module$ connecting multiple producers \newline
to one consumer, {\bf prioritizing lower producers} \newline
\verb$RRArbiter$ is a \verb$Module$ connecting multiple producers \newline
to one consumer {\bf in round-robin order} \newline
\subsubsection{Constructor}
\verb$val arb = Arbiter(gen:Data, n:Int)$ \newline
\begin{tabular}{l l l}
& \verb$gen$ & data type \\
& \verb$n$ & number of producers \\
\end{tabular}
\subsubsection{Interface}
\begin{tabular}{l l l}
& \verb$.io.in$ & \verb$Vec$ of \verb$Decoupled$ inputs \\
& \verb$.io.out$ & output as \verb$Decoupled$ \\
\end{tabular}

\end{multicols}
\end{document}
